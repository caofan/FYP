\documentclass{article}
\usepackage{amsmath}
\usepackage{amsthm}
\usepackage{amssymb}
\usepackage{algorithmic}
\newtheorem{theorem}{Theorem}[section]
\newtheorem{lemma}[theorem]{Lemma}
\newtheorem{proposition}[theorem]{Proposition}
\newtheorem{corollary}[theorem]{Corollary}
\newcommand{\Floor}[1]{\ensuremath{\lfloor #1\rfloor}}
\newcommand{\Ceil}[1]{\ensuremath{\lceil #1\rceil}}


\begin{document}

%Structure
\begin{figure}[h]
\begin{center}
\fbox{\parbox{\textwidth}{%

\noindent $ReadJunc$
\begin{algorithmic}[1]
\STATE $tuple(p,q,r)\ Key$;
\STATE $Set\{ReadJunc\}\ Nexts;$
\end{algorithmic}
}} %% end of fbox

\end{center}
\caption{Data structure $ReadJunc$ for storing alignments.}
\label{datastruct:readjuncs}
\end{figure}

\section{Data structure}
$ReadJunc$ as shown in Figure-\ref{datastruct:readjuncs} is a recursive data structure used to store the alignments found by $Find_All_Juncs$. It has two components, the $key$ and a set containing its descendants. The $key$ is a tuple of three elements, $p, q and r$, where $(p,q)$ define the starting and ending coordinates of the junction in the reference genome, respectively. The $p$ component is 0-based and the $q$ component is 1-based. Component $r$ is the size of the left anchor of the junction.


%Find_All_Junc and Seek_Junc
\begin{figure}[h]
\begin{center}
\fbox{\parbox{\textwidth}{%

\noindent $Find\_All\_Junc(R,\Phi=(G,l,L_{min},k,D_1..D_2))$
\begin{algorithmic}[1]
\STATE $Min\_Juncs=\infty$;
\STATE $ReadJunc\ H$;
\STATE $H.Key=(0,0,0)$;
\STATE $H.Nexts=Seek\_Junc(R,1,\Phi,\{[1,|G|]\},0)$;
\STATE $Transcripts=\emptyset$;
\STATE $Enum\_Transcripts(H,T,0,Min\_Junc,Transcripts)$
\STATE $Remove\_Duplicates(Transcripts)$;
\end{algorithmic}
}} %% end of fbox

\fbox{\parbox{\textwidth}{%
\noindent $Seek\_Junc(S,a,\Phi,X,L)$
\begin{algorithmic}[1]
\STATE $Juncs=\emptyset;T_1=T_2=A=\emptyset$;
\IF {$a=|S|$}
	\STATE $Min\_Junc=L$;
	\RETURN $Juncs$;
\ENDIF

%extend exon..
\STATE $A=Extend(S[a,\min(a+l,|S|)],X,\Phi)$
\IF {$A\ne\emptyset$}
	\STATE $Juncs=Seek\_Junc(S,\min(a+l,|S|),\Phi,A,L)$;
\ENDIF
\IF {$a>1$ and $|S|-a>l$}
	\IF {$X$ contains only genomic locations}
		\STATE $X=X-a$;
	\ELSE
		\STATE $X=\emptyset$
	\ENDIF
	\IF {$L<Min\_Junc$ }
	%Search for single junctions: junction is in S[a,a+l-1]$..
		\STATE $T_1=Find\_Single\_Junc(S[1,\min(a+2l-1,|S|)],\Phi,X)$;
		\FOR {$(p,q,r)\in T_1$}
			\STATE $ReadJunc H$;
			\STATE $ReadJunc H.key=(p,q,r)$;
			\STATE $H.Nexts=Seek\_Junc(S[r+1,|S|],1,\Phi,\{q\},L+1)$;
			\STATE $Hits.add(H)$;
		\ENDFOR
	\ENDIF
	\IF {$L+1<Min\_Junc$ and $L_{min}<2l$}
	%Search for double junctions..
		\FOR {$i=1,..,2l-L_{min}-1$ \bf{step} $L_{min}-l$}
			\STATE $T_2=T_2\cup Find\_Single\_Junc(S[1,\min(a+(L_{min}-l)+i,|S|)],\Phi,X)$;
		\ENDFOR
		\FOR {$(p,q,r)\in T_2$}
			\STATE $ReadJunc H$;
			\STATE $ReadJunc H.key=(p,q,r)$;
			\STATE $H.Nexts=Seek\_Junc(S[r+1,|S|],1,\Phi,\{q\},L+1)$;
			\STATE $Hits.add(H)$;
		\ENDFOR
	\ENDIF
\ENDIF
\IF {$A=\emptyset$ and $T_1=\emptyset$ and $T_2=\emptyset$}
	\RETURN $NULL$;
\ENDIF

\RETURN $Juncs$;
\end{algorithmic}
}} %% end of fbox

\end{center}
\caption{Algorithms $Find\_All\_Junc$ and $Seek\_Junc$.}
\label{algo:all_junc}
\end{figure}


%Enumerate Transcripts.
\begin{figure}[h]
\begin{center}
\fbox{\parbox{\textwidth}{%

\noindent $Enum\_Transcripts(H,T,L,Min\_Junc,Transcripts)$
\begin{algorithmic}[1]
\IF {$L>Min\_Junc$ or $H.Nexts=NULL$}
	\RETURN
\ENDIF
\IF {$H.Nexts=\emptyset$}
	\IF {$L=Min\_Junc$}
		\STATE store $T$ in $Transcripts$;
	\ENDIF
	\RETURN;
\ENDIF
\FOR {each $J$ in $H.Nexts$}
	\STATE $T[L]=J.Key$;
	\STATE $Enum\_Transcripts(J,T,L+1,Min\_Junc,Transcripts)$;
\ENDFOR
\end{algorithmic}
}} %% end of fbox
\end{center}
\caption{Algorithm $Enum\_Transcripts$.}
\label{algo:enum_trans}
\end{figure}
\end{document}